%% This document has been formatted to meet the requirements of the British
%% Ecological Society (BES)
\documentclass[12pt]{article}

\usepackage{url}
\usepackage{geometry}
\usepackage[round]{natbib}
\usepackage{graphicx}
\geometry{a4paper}
\usepackage[T1]{fontenc}
\usepackage[utf8]{inputenc}
\usepackage{authblk}
\usepackage[running]{lineno}
\usepackage{setspace}
\usepackage{framed}   % for text boxes
\usepackage{booktabs} % For creating pretty tables
\usepackage{ctable} % Uses booktabs but gives us captions and footnotes
\usepackage{footmisc} % For fancy symbols used in the table footnotes
\usepackage[section]{placeins}  % To keep floats in their designated sections
% using the \FloatBarrier command

%% This package allows us to skip a line between paragraphs
\usepackage{parskip}

%% The amssymb and mathrsfs packages provide various useful mathematical symbols
\usepackage{amssymb}
\usepackage{mathrsfs}

%%The amsmath package allows me to write nice equations (\align* and \tag)
\usepackage{amsmath}

\doublespacing

%\title{Incorporating variability in simulations of seasonally forced phenology using integral projection models}
%
%\author[a]{Devin W. Goodsman\thanks{dgoodsman@lanl.gov}}
%\author[b]{Brian H. Aukema}
%\author[c]{Nate McDowell}
%\author[a]{Richard S. Middleton}
%\author[a]{Chonggang Xu\thanks{cxu@lanl.gov}}
%
%\affil[a]{Earth and Environmental Science Division, Los Alamos National Laboratory, Los Alamos, New Mexico, 87545, USA}
%\affil[b]{Department of Entomology, 432A Hodson Hall, 1980 Folwell Ave, St Paul, Minnesota, 55108, USA}
%\affil[c]{Pacific Northwest National Laboratory, Richland, Washington, USA}
%
%\renewcommand\Authands{ and }
%\date{*Corresponding author: dgoodsman@lanl.gov \\
%\dag Corresponding author: cxu@lanl.gov}

\begin{document}

\section*{Appendix S2: mountain pine beetle development rates}

\renewcommand{\thetable}{A2.\arabic{table}}
\setcounter{table}{0}

\FloatBarrier

The rate equation for mountain pine beetle development, as defined in \citet{Regniere2012a}, is 

\begin{align*}
r[T] &= \psi\bigg[\text{exp}(\omega(T-T_b)) - \bigg(\frac{T_m - T}{T_m - T_b}\bigg)\text{exp}(-\omega(T - T_b)/\Delta\text{b}) - \\
&\bigg(\frac{T - T_b}{T_m - T_b}\bigg)\text{exp}(\omega(T_m - T_b) - (T_m - T)/\Delta \text{m})\bigg], \tag{eqn A2.1}
\end{align*}

where $T$ is temperature ($^\circ$C) and the six parameters ($\psi$, $\omega$, $T_m$, $T_b$, $\Delta \text{b}$, $\Delta \text{m}$) vary between life stages.  The development rate is given by eqn A3.1 except when $T \geq T_m$ or when $T \leq T_b$. When $T \geq T_m$ or when $T \leq T_b$, $r[T] = 0$. The parameter definitions and units are given in Table \ref{Table A2.1} and the estimated parameter values for each life stage are given in Table \ref{Table A2.2}. An additional stage-specific parameter, $\sigma_s$, is estimated under the assumption that development rates in the population are log-normally distributed around the median given by eqn A2.1 as described in equation 13.

\ctable[
  cap = Beetle parameter defs,
  caption = {Parameter definitions and units for eqn A2.1 developed by \citet{Regniere2012a}.},
  label = Table A2.1,
  pos = t
  ]{lll}{
  }{
    \toprule
    parameter & definition & units\\
    \midrule
    $T_b$ & base temperature for development & $^\circ$C\\ 
    $\Delta$b & width of lower thermal boundary layer & $^\circ$C\\ 
    $T_m$ & maximum temperature for development & $^\circ$C\\
    $\Delta$m & width of upper thermal boundary layer & $^\circ$C\\  
    $\omega$ & low temperature acceleration rate & $(^\circ\text{C})^{-1}$\\ 
    $\psi$ & peak rate control parameter & (day)$^{-1}$\\  
    \bottomrule
  }

\ctable[
  cap = Beetle development,
  caption = {Mountain pine beetle development rate parameter values estimated by \citet{Regniere2012a}. Note that $\sigma$ corresponds to $\sigma_\epsilon$ in Table 4 and $\sigma_\delta$ in Table 5 of \citet{Regniere2012a}. Due to extreme sensitivity of the shape of the curve to parameter values, we have included many decimal places as recommended by Jacques R\'egni\`ere (personal communication).},
  label = Table A2.2,
  pos = t
  ]{lllllllll}{
  }{
    \toprule
    parameter & oviposition & egg & L1, & L2 & L3 & L4, & pupae & teneral\\
    \midrule
    $T_b$ & 4.6341 & 7.0000 & 3.5559 & 6.9598 & 6.8462 & 16.2464 & 5.6300 & 4.2400\\ 
    $\Delta$b & 0.1 & 0.01930 & 0.10 & 0.09709 & 0.10 & 0.03905 & 0.10989 & 0.09997\\ 
    $T_m$ & 27.7587 & 30.0928 & 29.2647 & 28.9047 & 28.7013 & 28.0000 & 28.5500 & 35.0000\\ 
    $\Delta$m & 3.0759 & 4.4175 & 3.8227 & 3.0374 & 2.5359 & 4.5504 & 2.8600 & 7.1479\\  
    $\omega$ & 0.3684 & 0.2563 & 0.2398 & 0.3714 & 0.4399 & 0.2593 & 0.1532 & 0.1463\\ 
    $\psi$ & 0.005199 & 0.02317 & 0.01082 & 0.01072 & 0.003892 & 0.05034 & 0.02054 & 0.01173\\  
    $\sigma$ & 0.2458 & 0.1799 & 0.2911 & 0.3799 & 0.3868 & 0.3932 & 0.2998 & 0.5284\\ 
    \bottomrule
  }

\bibliographystyle{mee}

\bibliography{SPEcology}% your .bib file(s)

\end{document}